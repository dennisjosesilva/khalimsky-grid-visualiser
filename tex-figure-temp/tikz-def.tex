\usepackage{tikz}
\usetikzlibrary{arrows,shapes,backgrounds,shapes.geometric}
\usetikzlibrary{trees}
\usetikzlibrary{positioning}
\usetikzlibrary{calc}

\usepackage{arrayjobx}		% for grid drawing Tim Bombadil snippet
\usepackage{trimspaces}		% for grid drawing Tim Bombadil snippet
\usepackage{xifthen}		% for grid drawing Tim Bombadil snippet\pgfplotsset{compat=newest}

\tikzset{hasseNode/.style={draw,circle,inner sep=2pt,outer sep=0pt} }
\tikzset{hasseNodeG/.style={black!50,fill=black!50,draw,circle,inner sep=2pt,outer sep=0pt} }

\usetikzlibrary{shapes}

%Define block styles
\tikzstyle{block} = [rectangle, draw, fill=blue!20, 
text width=4.3em, text centered, rounded corners, minimum height=2.5em, node distance=1cm]
\tikzstyle{block2} = [rectangle, fill=white, 
text width=2.7em, minimum height=2.5em, node distance=1cm]
\tikzstyle{cloud} = [draw, ellipse,fill=red!20, node distance=1cm,
minimum height=2em]
\tikzstyle{init} = [pin edge={to-,thin,black}]

% % % % % % % % %TREE % % % % % % % % % %
% Set the overall layout of the tree
\tikzstyle{level 1}=[level distance=3.5cm, sibling distance=3.5cm]
\tikzstyle{level 2}=[level distance=3.5cm, sibling distance=2cm]

% Define styles for bags and leafs
\tikzstyle{bag} = [rectangle, draw, fill=blue!20, 
text width=4.3em, text centered, rounded corners, minimum height=2.5em, node distance=1cm]
\tikzstyle{end} = [circle, minimum width=2pt,fill, inner sep=0pt]

\tikzstyle{pnode}=[circle,draw,inner sep=0pt,minimum size=5mm]
\tikzstyle{qnode}=[rectangle,draw,inner sep=0pt,minimum size=3mm]
\tikzstyle{rnode}=[pnode]
\tikzstyle{anynode}=[diamond,draw,inner sep=0pt,minimum size=4mm]

\usetikzlibrary{arrows.meta}

% Diagram
\tikzstyle{Process} = [rectangle, rounded corners, minimum width=3cm, minimum height=1cm, text centered, 
draw=black]

\tikzstyle{InputOutput} =  [text centered]

\tikzstyle{Arrow} =       [thick, ->, >=stealth]
\tikzstyle{DashedArrow} = [thick, ->, >=stealth, dashed]

% Interval
\usepackage{interval}

% Charles' graphic 
\usepackage{slashed}            % for slashed characters in math mode
%\usepackage{amsthm}             % For theorem environment
\usepackage{tikz} % tikz 
\usetikzlibrary{shapes,calc,matrix,arrows.meta,3d,quotes,positioning,fit}

\newcommand{\mygrid}{\draw[step=1] (0,0)  grid (3,3);}
\newcommand{\rowvector}{\draw[step=1] (0,0)  grid (8,1);}
\newcommand{\colvector}{\draw[step=1] (0,0)  grid (1,9);}


\makeatletter
\def\trimspace#1{\trim@spaces@in{#1}}
\makeatother

\newcommand{\drawBinaryImage}[3]% dataarray
{   
    \pgfmathtruncatemacro{\gridwidth}{#1}
    \pgfmathtruncatemacro{\gridheight}{#2}
    \dataheight=\gridwidth
    
    \draw[gray,densely dashed] (0,0) grid (\gridwidth,\gridheight);
    \foreach \x in {1,...,\gridwidth}
    { \foreach \y in {1,...,\gridheight}
        {   
            \pgfmathtruncatemacro{\colnum}{\x}
            \pgfmathtruncatemacro{\rownum}{\gridheight+1-\y}
            \expandafter\csname check#3\endcsname(\rownum,\colnum)\trimspace\cachedata
            \ifthenelse{\cachedata=0}
                {\fill[white,fill opacity=1] (\x-0.9,\y-0.9) rectangle (\x-0.1,\y-0.1);}
                {}
            \ifthenelse{\cachedata=1}
                {\fill[black,fill opacity=1]  (\x-0.9,\y-0.9) rectangle (\x-0.1,\y-0.1);}
                {}
            \ifthenelse{\cachedata=2}
                {\fill[blue,fill opacity=1] (\x-0.9,\y-0.9) rectangle (\x-0.1,\y-0.1);}
                {}
			\ifthenelse{\cachedata=3}
                {\fill[red,fill opacity=1] (\x-0.9,\y-0.9) rectangle (\x-0.1,\y-0.1);}
                {}
			\ifthenelse{\cachedata=4}
                {\fill[green,fill opacity=1] (\x-0.9,\y-0.9) rectangle (\x-0.1,\y-0.1);}
                {}
			\ifthenelse{\cachedata=5}
                {\fill[cyan,fill opacity=1] (\x-0.9,\y-0.9) rectangle (\x-0.1,\y-0.1);}
                {}
			\ifthenelse{\cachedata=6}
                {\fill[violet,fill opacity=1] (\x-0.9,\y-0.9) rectangle (\x-0.1,\y-0.1);}
                {}
			\ifthenelse{\cachedata=7}
                {\fill[orange,fill opacity=1] (\x-0.9,\y-0.9) rectangle (\x-0.1,\y-0.1);}
                {}
			\ifthenelse{\cachedata=8}
                {\fill[yellow,fill opacity=1] (\x-0.9,\y-0.9) rectangle (\x-0.1,\y-0.1);}
                {}
			\ifthenelse{\cachedata=9}
                {\fill[gray,fill opacity=1] (\x-0.9,\y-0.9) rectangle (\x-0.1,\y-0.1);}
                {}                
			\ifthenelse{\cachedata=10}
                {\fill[fill=blue!35] (\x-0.9,\y-0.9) rectangle (\x-0.1,\y-0.1);}
                {}                

        }
    }
}



\newcommand{\drawGrayImage}[3]% dataarray
{   
	\pgfmathtruncatemacro{\gridwidth}{#1}
	\pgfmathtruncatemacro{\gridheight}{#2}
	\dataheight=\gridwidth
	
	\draw[gray,densely dashed] (0,0) grid (\gridwidth,\gridheight);
	\foreach \x in {1,...,\gridwidth}
	{ \foreach \y in {1,...,\gridheight}
		{   
			\pgfmathtruncatemacro{\colnum}{\x}
			\pgfmathtruncatemacro{\rownum}{\gridheight+1-\y}
			\expandafter\csname check#3\endcsname(\rownum,\colnum)\trimspace\cachedata
			
			\ifthenelse{\cachedata=0}
			{
				\fill[black,fill opacity=1] (\x-0.9,\y-0.9) rectangle (\x-0.1,\y-0.1);        
				\draw[white, fill opacity=1] (\x-0.5,\y-0.5)  node[] {$\cachedata$}; 
			}
			{}
			
			\ifthenelse{\cachedata=1}
			{
				\fill[black,fill opacity=0.85] (\x-0.9,\y-0.9) rectangle (\x-0.1,\y-0.1);        
				\draw[white, fill opacity=1] (\x-0.5,\y-0.5)  node[] {$\cachedata$};
			}
			{}
			
			\ifthenelse{\cachedata=2}
			{
				\fill[black,fill opacity=0.7] (\x-0.9,\y-0.9) rectangle (\x-0.1,\y-0.1);        
				\draw[white, fill opacity=1] (\x-0.5,\y-0.5)  node[] {$\cachedata$};
			}
			{}
			
			\ifthenelse{\cachedata=3}
			{
				\fill[black,fill opacity=0.55] (\x-0.9,\y-0.9) rectangle (\x-0.1,\y-0.1);        
				\draw[white, fill opacity=1] (\x-0.5,\y-0.5)  node[] {$\cachedata$};
			}
			{}
			
			\ifthenelse{\cachedata=4}
			{
				\fill[black,fill opacity=0.25] (\x-0.9,\y-0.9) rectangle (\x-0.1,\y-0.1);     
				\draw[black, fill opacity=1] (\x-0.5,\y-0.5)  node[] {$\cachedata$};   
			}
			{}
			
			\ifthenelse{\cachedata=5}
			{
				\fill[black,fill opacity=0.2] (\x-0.9,\y-0.9) rectangle (\x-0.1,\y-0.1);        
				\draw[black, fill opacity=1] (\x-0.5,\y-0.5)  node[] {$\cachedata$};   
			}
			{}
			
			\ifthenelse{\cachedata=6}
			{
				\fill[black,fill opacity=0.1] (\x-0.9,\y-0.9) rectangle (\x-0.1,\y-0.1);        
				\draw[black, fill opacity=1] (\x-0.5,\y-0.5)  node[] {$\cachedata$};   
			}
			{}
			
			\ifthenelse{\cachedata=7}
			{
				\fill[black,fill opacity=0] (\x-0.9,\y-0.9) rectangle (\x-0.1,\y-0.1);        
				\draw[black, fill opacity=1] (\x-0.5,\y-0.5)  node[] {$\cachedata$};   
			}
			{}
			
			
			\ifthenelse{\cachedata=11}
			{
				\fill[black,fill opacity=0.85] (\x-0.9,\y-0.9) rectangle (\x-0.1,\y-0.1);        
				\draw[white, fill opacity=1] (\x-0.5,\y-0.5)  node[] {$1$};
			}
			{}
			
			\ifthenelse{\cachedata=2}
			{
				\fill[black,fill opacity=0.7] (\x-0.9,\y-0.9) rectangle (\x-0.1,\y-0.1);        
				\draw[white, fill opacity=1] (\x-0.5,\y-0.5)  node[] {$\cachedata$};
			}
			{}
			
			\ifthenelse{\cachedata=3}
			{
				\fill[black,fill opacity=0.55] (\x-0.9,\y-0.9) rectangle (\x-0.1,\y-0.1);        
				\draw[white, fill opacity=1] (\x-0.5,\y-0.5)  node[] {$\cachedata$};
			}
			{}
			
			\ifthenelse{\cachedata=4}
			{
				\fill[black,fill opacity=0.4] (\x-0.9,\y-0.9) rectangle (\x-0.1,\y-0.1);     
				\draw[black, fill opacity=1] (\x-0.5,\y-0.5)  node[] {$\cachedata$};   
			}
			{}
			
			\ifthenelse{\cachedata=5}
			{
				\fill[black,fill opacity=0.25] (\x-0.9,\y-0.9) rectangle (\x-0.1,\y-0.1);        
				\draw[black, fill opacity=1] (\x-0.5,\y-0.5)  node[] {$\cachedata$};   
			}
			{}
			
			\ifthenelse{\cachedata=6}
			{
				\fill[black,fill opacity=0.1] (\x-0.9,\y-0.9) rectangle (\x-0.1,\y-0.1);        
				\draw[black, fill opacity=1] (\x-0.5,\y-0.5)  node[] {$\cachedata$};   
			}
			{}
			
			\ifthenelse{\cachedata=7}
			{
				\fill[black,fill opacity=0] (\x-0.9,\y-0.9) rectangle (\x-0.1,\y-0.1);        
				\draw[black, fill opacity=1] (\x-0.5,\y-0.5)  node[] {$\cachedata$};   
			}
			{}   
			
		}
	}
}

\newcommand{\drawCrossImage}[4]
{
   \pgfmathtruncatemacro{\x}{#3}
   \pgfmathtruncatemacro{\y}{#4}   

   \draw[#1,line width=#2] (\x+1, \y) -- (\x+2,\y) -- (\x+2,\y-1) -- (\x+3,\y-1) -- (\x+3,\y-2) --
                           (\x+2,\y-2) -- (\x+2, \y-3) -- (\x+1,\y-3) -- (\x+1,\y-2) -- (\x, \y-2) --
                           (\x, \y-1) -- (\x+1, \y-1) -- (\x+1, \y);
   
}

\newcommand{\drawCoordinateSystem}
{
    \node (y0) at (-0.5,6.5) {\tiny{$0$}};
    \node (y1) at (-0.5,5.5) {\tiny{$1$}};
    \node (y2) at (-0.5,4.5) {\tiny{$2$}};
    \node (y3) at (-0.5,3.5) {\tiny{$3$}};
    \node (y4) at (-0.5,2.5) {\tiny{$4$}};
    \node (y5) at (-0.5,1.5) {\tiny{$5$}};
    \node (y6) at (-0.5,0.5) {\tiny{$6$}};

    \node (x0) at (0.5,7.5) {\tiny{$0$}};
    \node (x1) at (1.5,7.5) {\tiny{$1$}};
    \node (x2) at (2.5,7.5) {\tiny{$2$}};
    \node (x3) at (3.5,7.5) {\tiny{$3$}};
    \node (x4) at (4.5,7.5) {\tiny{$4$}};
    \node (x5) at (5.5,7.5) {\tiny{$5$}};
    \node (x6) at (6.5,7.5) {\tiny{$6$}};

    \node (x) at (3.5, 8) {\small{$x$}};
    \node (y) at (-1, 3.5) {\small{$y$}};
    %
    \draw[thick, ->] (-0.2, 7.2) -- (-0.2,-0.5);
    \draw[thick, ->] (-0.2, 7.2) -- ( 7.5,  7.2);
}

\newcommand{\drawThreeXThreeBox}[4]
{
  \pgfmathtruncatemacro{\x}{#3}
  \pgfmathtruncatemacro{\y}{#4}
  
  \draw[#1,line width=#2] (\x, \y) rectangle (\x+3, \y-3);
}

\tikzset{gz/.style={fill=black, text=white}}
\tikzset{gf/.style={fill=black, opacity=0.25, text=black, text opacity=1}}
\newcommand{\drawImageGraph}
{
    \node[hasseNode,gz] (0) at (0,0) {\tiny{$0$}}; \node[hasseNode,gz] (1) at (1,0) {\tiny{$0$}}; 
    \node[hasseNode,gz] (2) at (2,0) {\tiny{$0$}}; \node[hasseNode,gz] (3) at (3,0) {\tiny{$0$}};  
    \node[hasseNode,gz] (4) at (4,0) {\tiny{$0$}}; \node[hasseNode,gz] (5) at (5,0) {\tiny{$0$}};
    \node[hasseNode,gz] (6) at (6,0) {\tiny{$0$}};

    \node[hasseNode,gz] (7) at (0,1) {\tiny{$0$}}; \node[hasseNode] (8) at (1,1) {\tiny{$7$}}; 
    \node[hasseNode] (9) at (2,1) {\tiny{$7$}}; \node[hasseNode,gf] (10) at (3,1) {\tiny{$4$}};  
    \node[hasseNode] (11) at (4,1) {\tiny{$7$}}; \node[hasseNode] (12) at (5,1) {\tiny{$7$}};
    \node[hasseNode] (13) at (6,1) {\tiny{$7$}};

    \node[hasseNode,gz] (14) at (0,2) {\tiny{$0$}}; \node[hasseNode,gf] (15) at (1,2) {\tiny{$4$}}; 
    \node[hasseNode,gf] (16) at (2,2) {\tiny{$4$}}; \node[hasseNode,gf] (17) at (3,2) {\tiny{$4$}}; 
    \node[hasseNode] (18) at (4,2) {\tiny{$7$}}; \node[hasseNode,gf] (19) at (5,2) {\tiny{$4$}};
    \node[hasseNode] (20) at (6,2) {\tiny{$7$}};

    \node[hasseNode,gz] (21) at (0,3) {\tiny{$0$}}; \node[hasseNode] (22) at (1,3) {\tiny{$7$}};
    \node[hasseNode,gf] (23) at (2,3) {\tiny{$4$}}; \node[hasseNode,gf] (24) at (3,3) {\tiny{$4$}}; 
    \node[hasseNode] (25) at (4,3) {\tiny{$7$}}; \node[hasseNode,gf] (26) at (5,3) {\tiny{$4$}};
    \node[hasseNode] (27) at (6,3) {\tiny{$7$}};

    \node[hasseNode,gz] (28) at (0,4) {\tiny{$0$}}; \node[hasseNode] (29) at (1,4) {\tiny{$7$}}; 
    \node[hasseNode] (30) at (2,4) {\tiny{$7$}}; \node[hasseNode,gf] (31) at (3,4) {\tiny{$4$}}; 
    \node[hasseNode] (32) at (4,4) {\tiny{$7$}}; \node[hasseNode,gf] (33) at (5,4) {\tiny{$4$}};
    \node[hasseNode] (34) at (6,4) {\tiny{$7$}};

    \node[hasseNode,gz] (35) at (0,5) {\tiny{$0$}}; \node[hasseNode,gf] (36) at (1,5) {\tiny{$4$}}; 
    \node[hasseNode,gf] (37) at (2,5) {\tiny{$4$}}; \node[hasseNode,gf] (38) at (3,5) {\tiny{$4$}};  
    \node[hasseNode] (39) at (4,5) {\tiny{$7$}}; \node[hasseNode] (40) at (5,5) {\tiny{$7$}};
    \node[hasseNode] (41) at (6,5) {\tiny{$7$}};

    \node[hasseNode,gz] (42) at (0,6) {\tiny{$0$}}; \node[hasseNode,gz] (43) at (1,6) {\tiny{$0$}}; 
    \node[hasseNode,gz] (44) at (2,6) {\tiny{$0$}}; \node[hasseNode,gz] (45) at (3,6) {\tiny{$0$}};  
    \node[hasseNode,gz] (46) at (4,6) {\tiny{$0$}}; \node[hasseNode,gz] (47) at (5,6) {\tiny{$0$}};
    \node[hasseNode,gz] (48) at (6,6) {\tiny{$0$}};

    \draw (0) -- (1) -- (2) -- (3) -- (4) -- (5) -- (6);
    \draw (7) -- (8) -- (9) -- (10) -- (11) -- (12) -- (13);
    \draw (14) -- (15) -- (16) -- (17) -- (18) -- (19) -- (20);
    \draw (21) -- (22) -- (23) -- (24) -- (25) -- (26) -- (27);
    \draw (28) -- (29) -- (30) -- (31) -- (32) -- (33) -- (34);
    \draw (35) -- (36) -- (37) -- (38) -- (39) -- (40) -- (41);
    \draw (42) -- (43) -- (44) -- (45) -- (46) -- (47) -- (48);
    
    \draw (0) -- (7) -- (14) -- (21) -- (28) -- (35) -- (42);
    \draw (1) -- (8) -- (15) -- (22) -- (29) -- (36) -- (43);
    \draw (2) -- (9) -- (16) -- (23) -- (30) -- (37) -- (44);
    \draw (3) -- (10) -- (17) -- (24) -- (31) -- (38) -- (45);
    \draw (4) -- (11) -- (18) -- (25) -- (32) -- (39) -- (46);
    \draw (5) -- (12) -- (19) -- (26) -- (33) -- (40) -- (47);
    \draw (6) -- (13) -- (20) -- (27) -- (34) -- (41) -- (48);

}

\newcommand{\drawPattern}[4]
{
    \draw[gray,densely dashed] (0,0) grid (2,2);

    \fill[white, fill opacity=1] (0.1,1.1) rectangle (0.9, 1.9);        
    \draw[black, fill opacity=1] (0.5,1.5) node[] {\tiny{#1}};

    \fill[white, fill opacity=1] (1.1,1.1) rectangle (1.9, 1.9);        
    \draw[black, fill opacity=1] (1.5,1.5) node[] {\tiny{#2}};

    \fill[white, fill opacity=1] (0.1,0.1) rectangle (0.9, 0.9);        
    \draw[black, fill opacity=1] (0.5,0.5) node[] {\tiny{#3}};

    \fill[white, fill opacity=1] (1.1,0.1) rectangle (1.9, 0.9);        
    \draw[black, fill opacity=1] (1.5,0.5) node[] {\tiny{#4}};
}


\newcommand{\drawRedChannel}[3]% dataarray
{   
	\pgfmathtruncatemacro{\gridwidth}{#1}
	\pgfmathtruncatemacro{\gridheight}{#2}
	\dataheight=\gridwidth
	
	\draw[gray,densely dashed] (0,0) grid (\gridwidth,\gridheight);
	\foreach \x in {1,...,\gridwidth}
	{ \foreach \y in {1,...,\gridheight}
		{   
			\pgfmathtruncatemacro{\colnum}{\x}
			\pgfmathtruncatemacro{\rownum}{\gridheight+1-\y}
			\expandafter\csname check#3\endcsname(\rownum,\colnum)\trimspace\cachedata
			
			\ifthenelse{\cachedata=0}
			{
				\fill[black, fill opacity=1] (\x-0.9,\y-0.9) rectangle (\x-0.1,\y-0.1);        
				\draw[white, fill opacity=1] (\x-0.5,\y-0.5)node[] {\scriptsize $\cachedata$}; 				
			}
			{
				\fill[red, fill opacity=0.143*\cachedata] (\x-0.9,\y-0.9) rectangle (\x-0.1,\y-0.1);        
				\draw[white, fill opacity=1] (\x-0.5,\y-0.5)  node[] {\scriptsize $\cachedata$}; 
			}			
		}
	}
}

\newcommand{\drawGreenChannel}[3]% dataarray
{   
	\pgfmathtruncatemacro{\gridwidth}{#1}
	\pgfmathtruncatemacro{\gridheight}{#2}
	\dataheight=\gridwidth
	
	\draw[gray,densely dashed] (0,0) grid (\gridwidth,\gridheight);
	\foreach \x in {1,...,\gridwidth}
	{ \foreach \y in {1,...,\gridheight}
		{   
			\pgfmathtruncatemacro{\colnum}{\x}
			\pgfmathtruncatemacro{\rownum}{\gridheight+1-\y}
			\expandafter\csname check#3\endcsname(\rownum,\colnum)\trimspace\cachedata
			
			\ifthenelse{\cachedata=0}
			{
				\fill[black, fill opacity=1] (\x-0.9,\y-0.9) rectangle (\x-0.1,\y-0.1);        
				\draw[white, fill opacity=1] (\x-0.5,\y-0.5)node[] {\scriptsize $\cachedata$}; 				
			}
			{
				\fill[green, fill opacity=0.143*\cachedata] (\x-0.9,\y-0.9) rectangle (\x-0.1,\y-0.1);        
				\draw[black, fill opacity=1] (\x-0.5,\y-0.5)  node[] {\scriptsize $\cachedata$}; 
			}			
		}
	}
}

\newcommand{\drawBlueChannel}[3]% dataarray
{   
	\pgfmathtruncatemacro{\gridwidth}{#1}
	\pgfmathtruncatemacro{\gridheight}{#2}
	\dataheight=\gridwidth
	
	\draw[gray,densely dashed] (0,0) grid (\gridwidth,\gridheight);
	\foreach \x in {1,...,\gridwidth}
	{ \foreach \y in {1,...,\gridheight}
		{   
			\pgfmathtruncatemacro{\colnum}{\x}
			\pgfmathtruncatemacro{\rownum}{\gridheight+1-\y}
			\expandafter\csname check#3\endcsname(\rownum,\colnum)\trimspace\cachedata
			
			\ifthenelse{\cachedata=0}
			{
				\fill[black, fill opacity=1] (\x-0.9,\y-0.9) rectangle (\x-0.1,\y-0.1);        
				\draw[white, fill opacity=1] (\x-0.5,\y-0.5)node[] {\scriptsize $\cachedata$}; 				
			}
			{
				\fill[blue, fill opacity=0.143*\cachedata] (\x-0.9,\y-0.9) rectangle (\x-0.1,\y-0.1);        
				\draw[white, fill opacity=1] (\x-0.5,\y-0.5)  node[] {\scriptsize $\cachedata$}; 
			}			
		}
	}
}

\newcommand{\drawContinuousBinaryImage}[3]%
{
	\pgfmathtruncatemacro{\gridwidth}{#1}
	\pgfmathtruncatemacro{\gridheight}{#2}
	\dataheight=\gridwidth

	\draw[gray, densely dashed] (0,0) grid (\gridwidth, \gridheight);
	\foreach \x in {1,...,\gridwidth} { 
		\foreach \y in {1,...,\gridheight} {
			\pgfmathtruncatemacro{\colnum}{\x}
			\pgfmathtruncatemacro{\rownum}{\gridheight+1-\y}
			\expandafter\csname check#3\endcsname(\rownum,\colnum)\trimspace\cachedata
			
			\ifthenelse{\cachedata=1}{
				\draw[black, line width=1.5pt] (\x-1,\y-1) rectangle (\x,\y);
				\fill[black, fill opacity=0.65] (\x-1,\y-1) rectangle (\x,\y);
				\fill[black] (\x-1,\y-1) circle (0.1);      \fill[black] (\x,\y-1) circle (0.1);
				\fill[black] (\x,\y) circle (0.1);  \fill[black] (\x-1,\y) circle (0.1);
			}{}
		}
	}
}

\makeatletter
\tikzset{circle split part fill/.style  args={#1,#2}{%
 alias=tmp@name, % Jake's idea !!
  postaction={%
    insert path={
     \pgfextra{% 
     \pgfpointdiff{\pgfpointanchor{\pgf@node@name}{center}}%
                  {\pgfpointanchor{\pgf@node@name}{east}}%            
     \pgfmathsetmacro\insiderad{\pgf@x}
      %\begin{scope}[on background layer]
      %\fill[#1] (\pgf@node@name.base) ([xshift=-\pgflinewidth]\pgf@node@name.east) arc
      %                    (0:180:\insiderad-0.5\pgflinewidth)--cycle;
      %\fill[#2] (\pgf@node@name.base) ([xshift=\pgflinewidth]\pgf@node@name.west)  arc
      %                     (180:360:\insiderad-0.5\pgflinewidth)--cycle;  
      \fill[#1] (\pgf@node@name.base) ([xshift=-\pgflinewidth]\pgf@node@name.east) arc
                          (0:180:\insiderad-\pgflinewidth)--cycle;
      \fill[#2] (\pgf@node@name.base) ([xshift=\pgflinewidth]\pgf@node@name.west)  arc
                           (180:360:\insiderad-\pgflinewidth)--cycle;            %  \end{scope}   
         }}}}}  
 \makeatother 